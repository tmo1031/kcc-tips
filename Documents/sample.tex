\documentclass[11pt,a4paper,uplatex]{jsarticle}

\usepackage[truedimen,top=48truemm,bottom=15truemm,hmargin=27truemm]{geometry}
\setlength\fullwidth{40zw}
\setlength\textwidth{\fullwidth}
\setlength{\evensidemargin}{\oddsidemargin}
\setlength\footskip{0truemm}

\usepackage{setspace}
\setstretch{2} % ページ全体の行間を設定
\pagestyle{plain}

%% natbib.sty を使う.
\usepackage{natbib}

\usepackage{longtable}
\usepackage{booktabs}

\usepackage[utf8]{inputenc}
\usepackage{atbegshi}
\AtBeginShipoutFirst{\special{pdf:tounicode  UTF8-UCS2}}

\usepackage[dvipdfmx]{hyperref}
\usepackage{pxjahyper}

\usepackage[american]{babel}
\usepackage{csquotes}
\usepackage{url}

\bibliographystyle{/data/bin/myjecon}

\usepackage{endnotesj}
\renewcommand{\footnote}{\endnote} % footnote を endnote として使う
\renewcommand{\notesname}{註} % Notes を註に変更する
\patchcmd{\enoteformat}{1.8em}{0pt}{}{}
\renewcommand{\bibname}{参考文献}
\renewcommand\refname{参考文献}
\renewcommand{\figurename}{図.}
\renewcommand{\tablename}{表.}
\def\tightlist{}

\usepackage{amssymb}

\title{20XX年 架空科目:論証型レポート}
\author{学籍番号XXXXXXXX 氏名}
\date{\today}

\begin{document}
\maketitle


\section*{序}
\addcontentsline{toc}{section}{序}

レポート課題で合格するにはどのようなレポートを書くと良いのだろうか?
この問いは私たち慶應通信生にとって避けて通れない課題である。
ある人は問いにこのように答えるかもしれない。
「レポート課題を数多く書いていくと、
自然に量から質への転換が起こり、コツを掴むことができる。
そうなれば我思うままに書いたものが即ち合格レポートである。」
これも一つの答えではあるが、もはや悟りの境地と言えよう。
入学直後でコツを掴めずに脱落していく通信生も数多くいることを鑑みるに、
量から質への転換は答えの一つではあるが、より洗練されたやり方が求められる。
以上のような問題意識から、本論では慶應通信で合格レポートを書くための要点を検討していく。

本論の構成は以下の通り。
まず、一般に合格レポートはどのような要件を満たすかを考える(1章)。
次に、具体的に問われている課題に対応するため、レポートの出題事例を分類・分析を行う(2章)。
そして、上記を綜合して合格レポートを書くための要点を検討する(3章)。
その結果、課題タイプの見極めと適切な章立て・論理構成の2つが重要であると結論する。

\section{合格レポートが満たすべき要件:章立て・内容・体裁}

まず、この章では合格レポートとはどのようなものかを検討していく。

一般にレポート課題は次の4種類に分類できる\footnote{\citet{totayama2002},
  p.54}。
まず、報告型の課題として、読んで報告するタイプ、調べて報告するタイプの2つがある。
他方、論証型の課題として、問題が与えられた上で論じるタイプ、問題を自分で立てて論じるタイプの2つがある。
この分類を踏まえた上で、レポートが評価される点を考えると、2つの異なったレベルの要件がある。
その2つとは、レポート一般に当てはまる評価基準と、慶應通信の評価基準である。

\subsection{レポート一般に当てはまる評価基準:章立て}

まず、レポート一般に当てはまる要件として、以下2つが挙げられる\footnote{\citet{kawano2018},
  p.32-42}。

\begin{itemize}
\tightlist
\item
  「問い」「論証」「答え」が明確にある
\item
  「序論」「本論」「結論」の構成を持つ
\end{itemize}

\noindent
この2つの要件は密接に関わり合い、「問い」は「序論」、「論証」は「本論」、「答え」は「結論」にそれぞれ対応する。
要件が適用される例外として、冒頭で分類した報告型と論証型のうち、報告型については「問い」の重要性が低いと考える。
その理由は、「問い」は既に明確に示されており、重要なのはいかに明確に「論証」「答え」を示せるかがレポートの評価点と言えるからである。
そのため、報告型の課題に関しては、明確な「序論」「本論」「結論」の構成を持たなくても差し支えないだろう。
一方、論証型の課題に関しては、「序論」「本論」「結論」の構成を持っていないと、レポートとしての評価は低いと予測される。

さらに、論証の型については、戸田山の分類が参考になる\footnote{\citet{totayama2002},
  p.169}。
戸田山によれば、論証には妥当な論証形式とちょっと弱い論証形式があり、
そのうち妥当な論証形式は演繹的な形式を持つという。
私たちがレポートで論証する場合に全てを演繹的な構造にすることは難しいかもしれないが、
論理的な繋がりを意識して構成することは参考になるだろう。
代表的な論証の構成として、演繹的に論を構成する場合の章立ての例を示す。

\begin{itemize}
\tightlist
\item
  序論:課題中のキーワードを特定し、重要性を位置付け、主張する内容を予告する
\item
  1章(一般論):キーワードについての一般論をテキスト、参考文献からまとめる
\item
  2章(具体論):個別具体のトピックを参考文献などから説明する
\item
  3章(比較・考察):一般論・具体論から導き出される帰結を述べ、考察する
\item
  結論:本論の内容を振り返り主張をくりかえす
\end{itemize}

\subsection{慶應通信の評価基準:内容・体裁}

次に、慶應通信の評価基準について、レポート添付の講評欄\footnote{\citet{keiouniv2013}}を参照する。
講評欄には、8個のチェック項目があり、これらの項目を全て、もしくはほとんどをクリアしているレポートが合格となる。
8つの項目をよく見ると、内容の評価と、体裁の評価の2つに分けられることに気づく。
分類し、整理したリストを以下に示す。

\begin{itemize}
\tightlist
\item
  内容

  \begin{itemize}
  \tightlist
  \item
    テキストを精読し、内容を十分に理解しているか?
  \item
    与えられた課題の趣旨を正しく把握しているか?
  \item
    自分なりの論旨が展開できているか?
  \item
    参考文献を十分に利用しているか?
  \end{itemize}
\item
  体裁

  \begin{itemize}
  \tightlist
  \item
    参考文献の引用や記載方法が正確であるか?
  \item
    レポートの文章表現が整っているか?
  \item
    字数や用紙の使用方法が正確であるか?
  \item
    誤字・脱字なく記述されているか?
  \end{itemize}
\end{itemize}

まず、このうち前半の4つの内容についての評価項目を検討していく。
先ほどの演繹的に論を構成する例で見ると、各章の役割は次のようになる。
テキストの理解を1章(一般論)で示し、 課題趣旨把握を序論で示し、
自分なりの論旨展開を3章(比較考察)で示し、
参考文献の利用を2章(具体論)で示す。
このようにそれぞれの章で必要な記述がなされていることが評価の対象となるだろう。
もちろん課題のタイプや論の構成によって、各章の役割は異なるため、
全てが例に示した対応がとれるわけではないことにも留意が必要である。

次に、後半の4つの内容についての評価項目を検討していく。
第1の参考文献の引用については様々な作法があるが \footnote{\citet{kawano2018},
  p.72-96} \footnote{\citet{sako2012}, p.91-108}、
このうちその科目に適切な作法を選択して、統一して引用、註、文献表を作成することが必要である。
第2のレポートの文章表現については、日本語表現として主語が必要であることはもちろんのこと、
文と文との流れや繋がりをわかりやすくするための接続詞を適切に使うことも必要である。
第3の字数・用紙に関しては、ほとんどのレポート課題は4000字程度の分量を課しているため、
先ほどの本論3章構成の場合、序論500字、各章1000字、結論500字程度の配分がバランスがよい。
第4の誤字・脱字に関しては、提出前によく見直して推敲を重ねることが必要である。

以上のように、合格レポートが満たすべき要件として、章立て・内容・体裁の重要性が確認できた。

\section{レポート課題の分析:課題の9つのタイプ}

次に、この章ではレポートで問われる具体的な課題を検討していく。
前章で提示した戸田山による4分類を踏まえて慶應通信のレポート課題をみてみると、
書評、説明、演習、郷土史、作品分析、一行問題、事例問題、指定論述、自由論述の9つタイプがみてとれる。
詳細は付録を参照のこと。
独自の小分類を行ったものと、一般的な4分類を対応させると次のようになる。
つまり、読んで報告するタイプとして書評があり、
調べて報告するタイプとして郷土史、演習、説明があり、
問題が与えられた上で論じるタイプとして作品分析、一行問題、事例問題、指定論述があり、
問題を自分で立てて論じるタイプとして自由論述がある。

さらに、2021年度のレポート課題集で出題形式を分類して集計を行うと、報告型が51、論証型が142である。
このように見ると、慶應通信のレポート課題の大多数(74%)が論証型である。

\clearpage

\begin{itemize}
\tightlist
\item
  報告型の課題

  \begin{itemize}
  \tightlist
  \item
    読んで報告するタイプ

    \begin{itemize}
    \tightlist
    \item
      書評
    \end{itemize}
  \item
    調べて報告するタイプ(実践を含む)

    \begin{itemize}
    \tightlist
    \item
      郷土史
    \item
      演習
    \item
      説明
    \end{itemize}
  \end{itemize}
\item
  論証型の課題

  \begin{itemize}
  \tightlist
  \item
    問題が与えられた上で論じるタイプ

    \begin{itemize}
    \tightlist
    \item
      作品分析
    \item
      一行問題
    \item
      事例問題
    \item
      指定論述
    \end{itemize}
  \item
    問題を自分で立てて論じるタイプ

    \begin{itemize}
    \tightlist
    \item
      自由論述 \clearpage
    \end{itemize}
  \end{itemize}
\end{itemize}

\section{合格レポートを書くための要点検討:見極めと章立て}

この章では、1章と2章で見てきたことから帰結される内容を確認し、吟味する。

1章では「レポートには評価基準があり、特に論証型の課題では「問い」「論証」「答え」に対応する章立てが必要である。」を確認した。
2章では「慶應通信で出題される課題は、大多数が論証型である。」を確認した。
これらの前提から帰結される結論は次の通りである。
「慶應通信で出題される課題の大多数には、適切な章立てが必要である。」
この帰結を吟味すると、少ないながらも論証型でない課題も存在することと、章立ての重要性の2つが、要点であると見てとれる。
2つの要点をそれぞれ確認していく。

まず、論証型でない課題に対応するには、回答する課題がどのタイプかを判別する必要がある。
つまり、見極めの力を磨くことが要点である。

他方、論証型の課題に対しては、 「問い」が把握されていること、
「論証」が適切であること、
「結論」が明快であること、が重要と考えられる。
つまり、適切な章立て・論理構成が要点である。
これを満たす論理構成として本論1章で例示した内容を繰り返しながらまとめると次のようになる。
序論で課題で何が問われているかを再定義、または論点を明確にする。
このときシラバスやテキストを読み込み、「どうしてその概念・キーワードが重要か」を把握していることが望ましい。
1章では大前提を示すため、テキストや参考文献の内容を理解していることを示す。
2章では小前提を示すため、具体的な事例を指定文献や参考文献の調査・分析を示す。
3章では自分なりの論を展開できていることを示すため、一般論と具体論を比較、考察し、その問題点や限界を示す。
結論では新たな議論を展開せず、本論の流れを振り返り、主張を繰り返す。

\section*{結論}
\addcontentsline{toc}{section}{結論}

以上のように本論では、慶應通信で合格レポートを書くための方法の要点を考察するため、
回答が満たすべき要件の検討、課題の分類、回答作成の取り組み方の順で検討を行ってきた。
その結果、まず一般にレポート課題は章立てが必要で、特に論証型の課題では「問い」「論証」「答え」に対応する章立てが必要であることを確認した。
次に慶應通信で出題される課題は、大多数が論証型の課題であることを確認した。
そしてこれらを綜合して考察した結果、
慶應通信で合格レポートを書くための要点として課題タイプの見極めと適切な章立て・論理構成の2つが重要であると結論した。
もちろんテキスト・参考文献の読み込みも重要ではあるが、以上の点を意識して課題へ取り組むことで、レポートの合格率が上がるだろうと考える。

計 3995字(表題部、註、文献表、付録を除く)

\clearpage

\section*{レポート課題の分類詳細}
\addcontentsline{toc}{section}{レポート課題の分類詳細}

\subsection*{書評}
\addcontentsline{toc}{subsection}{書評}

書評は2類科目で出題されるレポート形式である。出題文は次のような特徴を持っている。
「【指定図書】を読み、その内容を正確に要約した上で、著者の主張に対する自己の見解をまとめて、書評をしなさい。
」

\subsection*{郷土史}
\addcontentsline{toc}{subsection}{郷土史}

郷土史は2類科目で出題されるレポート形式である。出題文は次のような特徴を持っている。
「自分の住んでいる地域の歴史を調べ、それを日本全体の歴史の中に位置付けて論じなさい。」

\subsection*{演習}
\addcontentsline{toc}{subsection}{演習}

演習は主に総合教育科目の自然科学分野と経済学部専門科目で出題されるレポート形式である。出題文は数式を含む。
なお、書道の臨書についても幅広い意味では演習であるため、このタイプに分類している。

\subsection*{説明}
\addcontentsline{toc}{subsection}{説明}

説明は総合教育科目と専門教育科目どちらでも出題されるレポート形式である。出題文は次のようなキーワードを含む。
「説明せよ。」「調べなさい。」「まとめなさい。」

\subsection*{作品分析}
\addcontentsline{toc}{subsection}{作品分析}

作品分析は3類科目で出題されるレポート形式である。出題文は次のようなキーワードを含む。
「引用と分析を含めて、」「【特定の作品、作家、テーマ】について論じなさい。」「文学的な特色について論じなさい。」

\subsection*{一行問題}
\addcontentsline{toc}{subsection}{一行問題}

一行問題は法学部専門科目で出題されるレポート形式である。出題文は次のようなキーワードを含む。
「整理しつつ説明しなさい。」「検討しなさい。」「論じなさい。」

\subsection*{事例問題}
\addcontentsline{toc}{subsection}{事例問題}

事例問題は法学部専門科目で出題されるレポート形式である。出題文は次のようなキーワードを含む。
「以下の事例における、」「以上の事実関係を前提として、」

\subsection*{指定論述}
\addcontentsline{toc}{subsection}{指定論述}

指定論述は総合教育科目と専門教育科目どちらでも出題されるレポート形式である。出題文は次のようなキーワードを含む。
「論じよ。」「述べよ。」「考察せよ。」
指定論述は全課題のうち半数を占める。

\subsection*{自由論述}
\addcontentsline{toc}{subsection}{自由論述}

自由論述は主に1類科目で出題されるレポート形式である。出題文は次のようなキーワードを含む。
「関心を抱いたテーマについて、」「問題設定を行い、」「自由に論じよ。」

\clearpage

また、2021年度のレポート課題集から、出題形式を分類した結果を表に示す。

\begin{longtable}[]{@{}lrrrrrr@{}}
\caption{2021年度出題形式のタイプ別集計}\tabularnewline
\toprule\noalign{}
課題分類 & 総合教育 & 文第1類 & 文第2類 & 文第3類 & 経済学部 & 法学部 \\
\midrule\noalign{}
\endfirsthead
\toprule\noalign{}
課題分類 & 総合教育 & 文第1類 & 文第2類 & 文第3類 & 経済学部 & 法学部 \\
\midrule\noalign{}
\endhead
\bottomrule\noalign{}
\endlastfoot
書評 & & & 2 & & & \\
郷土史 & & & 2 & & & \\
演習 & 4 & 1 & & 1 & 6 & 2 \\
説明 & 8 & 4 & 2 & 5 & 7 & 7 \\
作品分析 & & & & 16 & & \\
一行問題 & & 1 & & & 5 & 17 \\
事例問題 & & & & & 1 & 6 \\
指定論述 & 9 & 5 & 16 & 19 & 14 & 20 \\
自由論述 & & 9 & & 2 & 1 & 1 \\
\end{longtable}

\theendnotes


\renewcommand\refname{参考文献}
\bibliography{sample.bib}


\end{document}
